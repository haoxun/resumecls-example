
\documentclass[zh,color]{resumecls}
% \usepackage{enumitem}

\name{詹昊恂}
% \organization{华南理工大学}
% \address{地址,邮编}
\mobile{+86 13570322748}
\mail{programmer.zhx@gmail.com}
\homepage{http://haoxun.org}
% \resumeurl{http://example.com/resume-zh.pdf}

\begin{document}

\begin{table}

\maketitle

%%%%%%%%%%%%%%%%%%%%%%%%%%%%%%%%%%%%%%%%%%%%%%%%%%%%%%%%%%%%%%%%%%%%%%%%%%%%%%%
\heading{教育经历}
\entry{2em}{Xrp{8em}}{
    \heiti{华南理工大学} & 广州 & 2011年9月至今\\
}
\entry{4em}{lXX}{
    本科 & 软件学院 & 软件工程专业 \\
}

%%%%%%%%%%%%%%%%%%%%%%%%%%%%%%%%%%%%%%%%%%%%%%%%%%%%%%%%%%%%%%%%%%%%%%%%%%%%%%%
\heading{综合素质}
\entry{2em}{lX}{
	信息获取 & 熟练使用Google,擅长使用英文进行信息检索;
	长期潜水于Stackoverflow、Quora与知乎,具备独立解决问题以及提出好问题的素养。\\

	语言 & 大量阅读英语技术资料,无阅读、书写障碍。熟悉普通话与粤语,表意清晰。\\

	文档编写 & 具备编写\LaTeX,Markdown与reStructuredText文档的能力。\\

}


%%%%%%%%%%%%%%%%%%%%%%%%%%%%%%%%%%%%%%%%%%%%%%%%%%%%%%%%%%%%%%%%%%%%%%%%%%%%%%%
\heading{技术相关}
\entry{2em}{lX}{
	Python & 仔细阅读过官方的Language Reference以及部分PEPs,对Python的语言机制有深入的了解;
	阅读过\emph{Python Cookbook}、\emph{Python 3 Object Oriented Programming}等书籍,熟悉Strandrad Library,熟悉Python开发中的常用Coding Convention;
	一年半开发经验,完成并发布的项目有GoogleTranslateAPI、MyGoogleDict、DocTemplate,正在进行GeekCMS与CLIPasswd项目的开发;
	熟悉开源框架Django,阅读过其部分源码,曾使用Django开发过一个课程项目ShareDoc;时常混迹于CPyUG(华蟒用户组),并积极参与技术问题的讨论。\\

	% Python(Intermediate Level) & 
	% \begin{itemize}[noitemsep,nolistsep]
	% 	\item 仔细阅读过官方的Language Reference以及部分PEPs,对Python的语言机制有深入的了解;
	% 	\item 阅读过\emph{Python Cookbook}、\emph{Python 3 Object Oriented Programming}等书籍,熟悉Strandrad Library,熟悉Python开发中的常用Coding Convention;
	% 	\item 一年半开发经验,完成并发布的项目有GoogleTranslateAPI、MyGoogleDict、DocTemplate,正在进行GeekCMS与CLIPasswd项目的开发;
	% 	\item 熟悉开源框架Django,阅读过其部分源码,曾使用Django开发过一个课程项目ShareDoc;时常混迹于CPyUG(华蟒用户组),并积极参与技术问题的讨论;有一定程度的代码洁癖。
	% \end{itemize}\\

	C++ & 阅读过\emph{C++ Primer},使用Qt框架开发过遗传算法的演示程序,了解常用数据结构、算法的C++实现。目前正在学习C++11的特性。\\

	算法 & 正在阅读\emph{CLRS},计划在2014年内做完所有习题。\\

	网络 & 阅读过\emph{Computer Networking: A Top-Down Approach},了解应用层与传输层协议。\\

	前端 & 曾经由于项目需要学习过前端相关技能,主要涉及JQuery与Bootstrap的API。\\

	*nix & 阅读过\emph{The Linux Command Line},熟悉CLI常用操作命令;\emph{OS X}长期用户,通过虚拟机安装使用\emph{CentOS(Minimal Installed)}。\\

	Vim & 长期使用Vim进行开发,熟悉常用命令与操作。\\

	Git & 长期使用Git进行项目的版本管理,Github重度用户。\\

}

%%%%%%%%%%%%%%%%%%%%%%%%%%%%%%%%%%%%%%%%%%%%%%%%%%%%%%%%%%%%%%%%%%%%%%%%%%%%%%%
\heading{个人项目}
\entry{2em}{lX}{
	GeekCMS & 用于生成静态网站的、plugin-based的系统与框架。本人个人站点是通过GeekCMS v0.3生成的。(Nov 23, 2013 - Apr 02, 2014)(v0.3 已发布)\\

	GoogleTranslateAPI & Google翻译服务API的Python封装。通过模拟translate.google.com前端的request数据包,实现翻译文本、获取文本TTS等服务。(Jan 23, 2014 - Feb 25, 2014)(v0.3 已发布)\\

	MyGoogleDict & Google翻译的命令行前端。提供文本翻译、TTS播放、获取查询记录等功能(Jan 18, 2014 - Feb 25, 2014)(v0.2.2 已发布)\\

	DocTemplate & 用来管理常用模板的小脚本。(Feb 17, 2014 - Feb 23, 2014)(v0.1 已发布)\\

	CLIPaswd & 个人密码管理工具。(Nov 22, 2013 - Now)(v0.1 进行中)\\

}

%%%%%%%%%%%%%%%%%%%%%%%%%%%%%%%%%%%%%%%%%%%%%%%%%%%%%%%%%%%%%%%%%%%%%%%%%%%%%%%
% \heading{科研经历}
% \entry{2em}{Xp{8em}}{
%     \heiti{地点} & 起止时间 \\
% }
% \entry{4em}{X}{实验室名称 \quad 职位}
% \entry{6em}{X}{
%     研究方向和具体内容 \\
%     发表成果(亦可使用bibtex,像这样\cite{label},见文档最后注释内容) \\
% }

%%%%%%%%%%%%%%%%%%%%%%%%%%%%%%%%%%%%%%%%%%%%%%%%%%%%%%%%%%%%%%%%%%%%%%%%%%%%%%%
% \heading{工作经历}
% \entry{2em}{Xp{8em}}{
%     \heiti{单位名称} & 起止时间 \\
% }
% \entry{4em}{X}{部门 \quad 职位}
% \entry{6em}{X}{
%     负责的具体事项 \\
%     工作的具体内容 \\
% }

%%%%%%%%%%%%%%%%%%%%%%%%%%%%%%%%%%%%%%%%%%%%%%%%%%%%%%%%%%%%%%%%%%%%%%%%%%%%%%%
% \heading{校园经历}
% \entry{2em}{Xp{8em}}{
%     经历1 & 起止时间 \\
%     经历2 & 起止时间 \\
% }

%%%%%%%%%%%%%%%%%%%%%%%%%%%%%%%%%%%%%%%%%%%%%%%%%%%%%%%%%%%%%%%%%%%%%%%%%%%%%%%
% \heading{获得荣誉}
% \entry{2em}{Xr}{
%     荣誉1 & 颁发时间 \\
%     荣誉2 & 颁发时间 \\
% }

%%%%%%%%%%%%%%%%%%%%%%%%%%%%%%%%%%%%%%%%%%%%%%%%%%%%%%%%%%%%%%%%%%%%%%%%%%%%%%%
% 如果不需要发表成果,注释这一段即可
% \heading{附:发表成果}
% \vspace{-6em}
% \bibliography{resume}

%%%%%%%%%%%%%%%%%%%%%%%%%%%%%%%%%%%%%%%%%%%%%%%%%%%%%%%%%%%%%%%%%%%%%%%%%%%%%%%
\end{table}
\end{document}
