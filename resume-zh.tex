
\documentclass[zh,color]{resumecls}
% \usepackage{enumitem}

\name{詹昊恂}
% \organization{华南理工大学}
% \address{地址,邮编}
\mobile{+86 13570322748}
\mail{programmer.zhx@gmail.com}
\homepage{http://haoxun.org}
% \resumeurl{http://example.com/resume-zh.pdf}

\begin{document}

\begin{table}

\maketitle

%%%%%%%%%%%%%%%%%%%%%%%%%%%%%%%%%%%%%%%%%%%%%%%%%%%%%%%%%%%%%%%%%%%%%%%%%%%%%%%
\heading{教育经历}
\entry{2em}{Xrp{8em}}{
    \heiti{华南理工大学} & Sep. 2011 - Jul. 2015\\
}
\entry{4em}{lXX}{
    本科 & 软件学院 & 软件工程专业 \\
}

% \heading{工作经历}
% \entry{2em}{Xp{8em}}{
%     \heiti{腾讯} & 起止时间 \\
% }
% \entry{4em}{X}{部门 \quad 职位}
% \entry{6em}{X}{
%     负责的具体事项 \\
%     工作的具体内容 \\
% }

%%%%%%%%%%%%%%%%%%%%%%%%%%%%%%%%%%%%%%%%%%%%%%%%%%%%%%%%%%%%%%%%%%%%%%%%%%%%%%%
\heading{从业经历}

\entry{2em}{Xrp{8em}}{
    \heiti{腾讯} & Jul. 2014 - Sep. 2014\\
}
\entry{4em}{lX}{
    信息安全部实习生 & 参与名为“新浪微博热门话题挖掘”的实习生项目。\\
}
%%%%%%%%%%%%%%%%%%%%%%%%%%%%%%%%%%%%%%%%%%%%%%%%%%%%%%%%%%%%%%%%%%%%%%%%%%%%%%%

%%%%%%%%%%%%%%%%%%%%%%%%%%%%%%%%%%%%%%%%%%%%%%%%%%%%%%%%%%%%%%%%%%%%%%%%%%%%%%%
% \heading{综合素质}
% \entry{2em}{lX}{
% 	信息获取 & 熟练使用Google,擅长使用英文进行信息检索,具备独立解决问题的能力;\\
% 
% 	语言 & 大量阅读英语技术资料,无阅读、书写障碍。熟悉普通话与粤语,表意清晰。\\
% 
% 	文档编写 & 具备编写\LaTeX,Markdown与reStructuredText文档的能力。\\
% }


%%%%%%%%%%%%%%%%%%%%%%%%%%%%%%%%%%%%%%%%%%%%%%%%%%%%%%%%%%%%%%%%%%%%%%%%%%%%%%%
\heading{技术相关}
\entry{2em}{lX}{
	Python & 仔细阅读过官方的Language Reference以及部分PEPs,对Python的语言机制有深入的了解;
	阅读过\emph{Python Cookbook}等书籍,熟悉Standard Library,熟悉开发中的常用Coding Convention;
	一年半开发经验,完成并发布的项目有GeekCMS、GoogleTranslateAPI、MyGoogleDict等;
	CPyUG长期潜水用户,Pycoder's Weekly订阅者,关注Python领域知识。\\

	C++ & 仔细阅读过\emph{C++ Primer 5ed},了解基本的C++11语法特性;阅读并实践过\emph{Google C++ Style Guide},具备编码规范意识;
	% 正在阅读\emph{The C++ Standard Library}与libstdc++、libc++源码,尝试把握标准库的细节与设计思路;
	了解与C++相关工具链、autotool、测试框架与文档工具;
	% 缺乏Metaprogramming的能力,计划通过学习函数式编程与阅读\emph{C++ Template}弥补此项缺陷;
	作为主程完成过一个基于Qrobot的聊天机器人项目,与一个腾讯的实习生项目。\\

	OS & 阅读过\emph{The Linux Command Line},了解CLI常用操作命令;\emph{OS X}、\emph{Ubuntu Server}长期用户。\\

	Vim & 长期使用Vim进行开发,熟悉常用命令与操作。\\

	Git & 长期使用Git进行项目的版本管理,Github重度用户。\\

	% 前端 & 曾经由于项目需要学习过前端相关技能,主要涉及JQuery与Bootstrap的API。\\

}
\end{table}

\newpage
%%%%%%%%%%%%%%%%%%%%%%%%%%%%%%%%%%%%%%%%%%%%%%%%%%%%%%%%%%%%%%%%%%%%%%%%%%%%%%%
\begin{table}

\heading{个人项目}
\entry{2em}{lX}{
	GeekCMS & 用于生成静态网站的、plugin-based的框架。我的个人站点是通过GeekCMS生成的。(Nov. 2013 - Apr. 2014)\\

	GoogleTranslateAPI & Google翻译服务API的Python封装。通过模拟translate.google.com前端的数据包交互,实现翻译文本、TTS等服务。(Jan. 2014 - Feb. 2014)\\

	MyGoogleDict & Google翻译的命令行前端。提供文本翻译、TTS播放、获取查询记录等功能(Jan. 2014 - Feb. 2014)\\

	DocTemplate & 用来管理常用模板的脚本工具。(Feb. 2014 - Feb. 2014)\\
}

\heading{团队项目}
\entry{2em}{lX}{
	ShareDoc & 基于Django的文件共享平台。(Sep. 2013 - Dec. 2013)\\

	Qrobot聊天机器人 & 基于Qrobot的聊天机器人项目。项目涉及音频输入输出、语音识别、中文分词、AIML规则匹配、语音合成等功能。
	本人作为主力程序员参与了项目的需求分析与设计过程,提供了技术清单,制定了开发模式,负责实现音频输入输出模块、Utility模块、AIML dispatcher等。(Jun. 2014 - Jul. 2014)\\

	新浪微博热门话题挖掘 & 这是腾讯的实习生项目。整个项目分为5个模块:数据落地、过滤垃圾微博、话题聚类、子话题挖掘与Web化呈现。我负责数据落地与子话题挖掘模块的开发。
	在项目开展中,我作为主程制定了开发细节(涉及工具链、autotool、测试框架),推广了编码规范。
	数据落地采用的是新浪的API,每天可以落地300万条微博,覆盖率约为3\%;子话题挖掘主要采取层次聚类方法,聚出子话题后选取代表性微博。(Jul. 2014 - Sep. 2014)\\
}

\end{table}
%%%%%%%%%%%%%%%%%%%%%%%%%%%%%%%%%%%%%%%%%%%%%%%%%%%%%%%%%%%%%%%%%%%%%%%%%%%%%%%
% \heading{科研经历}
% \entry{2em}{Xp{8em}}{
%     \heiti{地点} & 起止时间 \\
% }
% \entry{4em}{X}{实验室名称 \quad 职位}
% \entry{6em}{X}{
%     研究方向和具体内容 \\
%     发表成果(亦可使用bibtex,像这样\cite{label},见文档最后注释内容) \\
% }

%%%%%%%%%%%%%%%%%%%%%%%%%%%%%%%%%%%%%%%%%%%%%%%%%%%%%%%%%%%%%%%%%%%%%%%%%%%%%%%
% \heading{工作经历}
% \entry{2em}{Xp{8em}}{
%     \heiti{单位名称} & 起止时间 \\
% }
% \entry{4em}{X}{部门 \quad 职位}
% \entry{6em}{X}{
%     负责的具体事项 \\
%     工作的具体内容 \\
% }

%%%%%%%%%%%%%%%%%%%%%%%%%%%%%%%%%%%%%%%%%%%%%%%%%%%%%%%%%%%%%%%%%%%%%%%%%%%%%%%
% \heading{校园经历}
% \entry{2em}{Xp{8em}}{
%     经历1 & 起止时间 \\
%     经历2 & 起止时间 \\
% }

%%%%%%%%%%%%%%%%%%%%%%%%%%%%%%%%%%%%%%%%%%%%%%%%%%%%%%%%%%%%%%%%%%%%%%%%%%%%%%%
% \heading{获得荣誉}
% \entry{2em}{Xr}{
%     荣誉1 & 颁发时间 \\
%     荣誉2 & 颁发时间 \\
% }

%%%%%%%%%%%%%%%%%%%%%%%%%%%%%%%%%%%%%%%%%%%%%%%%%%%%%%%%%%%%%%%%%%%%%%%%%%%%%%%
% 如果不需要发表成果,注释这一段即可
% \heading{附:发表成果}
% \vspace{-6em}
% \bibliography{resume}

%%%%%%%%%%%%%%%%%%%%%%%%%%%%%%%%%%%%%%%%%%%%%%%%%%%%%%%%%%%%%%%%%%%%%%%%%%%%%%%
\end{document}
